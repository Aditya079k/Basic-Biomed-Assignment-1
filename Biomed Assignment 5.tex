\documentclass[12pt]{article}
\usepackage{graphicx}

\begin{document}
\begin{titlepage}
\begin{center}

\large{NATIONAL INSTITUTE OF TECHNOLOGY, RAIPUR}\\
[1.5mm]
\begin{figure}[h]
\centering
\includegraphics[scale=.5]{NITRaipur_Logo}
\end{figure}
\line(1,0){300}\\
[0.25in]
\LARGE{\bfseries EMERGING TECHNOLOGIES IN HEALTHCARE}\\
[2mm]
\line(1,0){200}\\
[0.25in]
\large{\bfseries ASSIGNMENT 5}\\
{Basic Biomedical Engineering}\\
[0.75cm]
\large{Aditya Shrotriya}\\
{Roll number- 21111005}\\
{February 25, 2022}\\
[1cm]
\large{Under the supervision of:}\\
{Dr. Saurabh Gupta}

\end{center}
\end{titlepage}
\clearpage
\tableofcontents
\clearpage

\section{Introduction}

Healthcare technology is poised to reshape the industry as it provide more advanced and efficient patient care. Just as other industries have had to adapt and evolve as new technologies have emerged, healthcare organisation must keep up with healthcare tech trends not only to stay competitive but to also be able to provide the best possible patients outcome.

While there are many exciting new developments in the world of healthcare technology, here's a closer look at five that have tremendous potential to change the industry for the better.

\section{Artificial Intelligence}

AI is transforming the way healthcare organisation manage and draw insights from the incredible amount of scientific data and patient information that's available. AI can be used to create and customize treatment plans and medication options for patients in a much faster and precise way than human healthcare terms can do on their own. AI can also help in other ways, such as advancing the field of genomic medicine by analysing complex genetic information to determine the best the best course of care for individuals based on their DNA. The hope is that AI can one day improve diagnostic accuracy and even predict health outcomes.

\section{Electronic Health Records}

Emerging health information technology has made it possible to maintain health records in a centralised, cloud based portal, which provides health care providers have all the information they need at their fingertips, which can be crucial in the case of an emergency, if there is a language barrier, or if a patient is unable to communicate. This type of healthcare tech is ideal for when doctors from different hospitals or medical files or diagnosis to determine the most optimal way to treat their condition.

\section{Wearable Devices}

Today's smart-watches and other wearables do a lot more than a count steps. They can monitor heart rates, tracks sleep patterns, detect heart issue like atrial fabrication, take temperature, act as ECG and blood pressure monitors and more. Wearing these devices allows patients to monitor their own health, which can help identify potential problems. And they can also share the reporting and data with their physicians as needed.

\begin{figure}[h]
\centering
\includegraphics[scale=.2]{Wearable_Medical_Devices}
\caption{Diagnostic Wearable Medical Device}
\end{figure}

 Wearables can also be helpful to monitor post-surgical patients and track their vital signs. Beyond smart-watches, other wearable medical devices are coming to the market (or are already being used) that let patients and their healthcare providers monitor glucose levels, and measure hand movements in Parkinson's patients. In the future, other wearable technology may be embedded in eyeglasses, clothing, and other devices.                         

\section{Robotics}

The field of robotics has made great strides as well, making it a top healthcare tech trend. Medical robots can help surgeons perform very precise and targeted procedure and therapies. Though the doctors still control the surgery, robots take away the possibility of human errors and can potentially reduce infections. Healthcare robotics are also poised to take over clerical and routine tasks to free up nursing and other healthcare professionals to focus more on direct patients.

\section{3D Bioprinting}

The invention of 3D printing is another new technology in the healthcare industry tat is proving to be trans-formative. This new field of 3D Bioprinting enables physicians to print artificial limbs, organs, joint replacement parts, and bio tissue.

\begin{figure}[h]
\centering
\includegraphics[scale=.2]{3D Printing}
\caption{3D Bioprinting in Healthcare}
\end{figure}

 In addition, in the field of pharmacology, there are ongoing experiments for printing pills and other meditations. Lastly, 3D printers can also help create medical devices and surgical tools. 

\section{Conclusion}

The future of medicine and patient care will increasingly rely on health related technology, which is why healthcare organisations must embrace emerging healthcare technologies to stay relevant in the coming years. By exploring healthcare tech trends and becoming an early adopter of new innovations, healthcare providers can provide cutting-edge care.      


\end{document}