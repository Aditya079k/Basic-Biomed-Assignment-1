\documentclass[12pt]{article}
\usepackage{graphicx}

\begin{document}
\begin{titlepage}
\begin{center}

\large{NATIONAL INSTITUTE OF TECHNOLOGY, RAIPUR}\\
[1.5mm]
\begin{figure}[h]
\centering
\includegraphics[scale=.5]{NITRaipur_Logo}
\end{figure}
\line(1,0){300}\\
[0.25in]
\huge{\bfseries FUTURE OF HEALTHCARE}\\
[2mm]
\line(1,0){200}\\
[0.25in]
\large{\bfseries Assignment 3}\\
{Basic Biomedical Engineering}\\
[0.75cm]
\large{Aditya Shrotriya}\\
{Roll number- 21111005}\\
{February 11, 2022}\\
[1cm]
\large{Under the supervision of:}\\
{Dr. Saurabh Gupta}

\end{center}
\end{titlepage}
\clearpage
\tableofcontents
\clearpage 

\section{Introduction}

The pandemic undoubtedly accelerated technological advancement and adoption in healthcare. It's now easier and faster for patients to procure medical services outside of the traditional four walls of the medical establishment, enhancing convenience and accessibility for all.

Telehealth has made it possible for patients to receive care without an in-person office visit. In addition, remote patient monitoring is becoming more widely accepted. Having exponentially growth in  popularity throughout the pandemic, this now includes wearable technology with impressive capabilities, from remote monitoring of vitals to remote echocardiograms. If not for the pandemic, it probably would have taken the healthcare industry another decade to reach where it is today.

\section{Data Sharing}

One of the biggest challenges within healthcare today is the lack of data sharing between providers. Patient data and information are not routinely shared across providers, which can cause avoidable challenges, frustration, delays and potential harmful outcomes for patients. It can also create issues for health system. Secure, complaint sharing of patient data and information across medical providers will be one of the most important advancements in the coming decade.

Withholding patient data and information leads to the accrual of extraordinary amounts of unnecessary healthcare costs. This is due, in large part, to inessential or redundant medical labs and workshop being done because providers do not have access to the patient's full medical history. 

Further, the accrual of needless costs exacerbates one of the largest problems within the healthcare system today- that only the well-insured can access high-quality care. 

To help increase access to care and to improve the overall industry for both health system and patients, using advanced technology to enhance data sharing is critical.

\section{Transparency into the Medical Life Cycle}

Technological advancements and adoption will help increase transparency into every step of the medical life cycle for patients. For years, the persistent lack of transparency into specific service and costs have the massive challenges and misunderstanding. Unfortunately, it's all too common for patients to not fully know what medical services is being done, where it's being done, when it's being done and at what price point. 

\begin{figure}[ht]
\centering
\includegraphics[scale=.3]{AI in healthcare}
\caption{AI in healthcare}
\end{figure}

Technology can help bridge this gap, providing patients with straightforward and easy access to critical details and information from anywhere. In fact, just under two years ago, the US department of health and health services issued two ground breaking rules that provided patient with secure , increased access to healthcare data, enabling them to make better informed decisions.

Further, it's important that this information is presented to patients in a way that is accessible to everyone, including those who are undereducated, under insured and the elderly. A much needed step to help increase access to quality healthcare for all, heightened transparency into the medical life cycle is key.

\section{Technological Challenges}

The largest challenge the healthcare industry faces when it comes to adopting new technology is the initial error rate. Generally, new technological products require iteration before they're sufficient reliable. This iterative process can be painful, potentially resulting in inaccurate predictions and inappropriate recommendations.

To avoid this, technologist and clinicians must closely collaborate when rolling out new technology. Together, they will need to carefully test new tools and identify fail-safe methods until reliability is sufficiently achieved.

\section{Conclusion}

It's evident that now it is the time for healthcare technologies to in still true, lasting change that will improve the industry for patients and provides nationwide.

This change will be made possible by the continued adoption of innovative technology combined with the rise of a mobile workforce. With certain medical employees not bound by the central physical location, more services can be given outside of the traditional medical establishment- including in a patient's neighbourhood or home.

Looking ahead, it's essential that that the healthcare industry remain focused on one common goal- ensuring that everyone, no matter personal circumstances, has access to high quality and highly affordable care. Advanced technology, made even more powerful by increased mobility, will make this a reality. 


\end{document}