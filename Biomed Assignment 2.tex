\documentclass[12pt]{article}
\usepackage{graphicx}


\begin{document}
\begin{titlepage}
\begin{center}

\large{NATIONAL INSTITUTE OF TECHNOLOGY, RAIPUR}\\
[1.5mm]
\begin{figure}[h]
\centering
\includegraphics[scale=.5]{NITRaipur_Logo}
\end{figure}
\line(1,0){300}\\
[0.25in]
\huge{\bfseries EVOLUTION OF MODERN HEALTHCARE SYSTEM}\\
[2mm]
\line(1,0){200}\\
[0.25in]
\large{\bfseries ASSIGNMENT 2}\\
{Basic Biomedical Engineering}\\
[0.75cm]
\large{Aditya Shrotriya}\\
{Roll number- 21111005}\\
{February 4, 2022}\\
[1cm]
\large{Under the supervision of:}\\
{Dr. Saurabh Gupta}

\end{center}
\end{titlepage}
\clearpage
\tableofcontents
\clearpage

\section{Introduction}

In the Information age, technology has transformed every part of our society , from the way we communicate to the way we work, and as the 21st century draws on, the same forces that are shaping our social and work environment are changing the way that the health care is delivered.

The rise of big data promises to fuel new treatments ans faster analysis of a range of diseases and public health issues, while new technology is changing every aspect of medical provision, from the way surgery and treatment is delivered to the collation and analysis of patient data through mobile and even wearable technology.  

\section{Wearable Devices}

There has been an explosion of wearable technology in recent years. By the summer of 2015, the health research institute was reporting that a fifth of US consumer owned a wearable technology product, and over 70 percent of consumers aged between 16 and 24 is said that they would be interested in owning a wearable device such as a smart band.

\begin{figure}[h]
\centering
\includegraphics[scale=0.4]{remote_diabetes_monitoring}
\caption{Remote Diabetes Monitoring}
\end{figure}

Health care apps and gadgets designed for wearable tech have the potential to help the 140 million people in the US who are living with atleast one long term medical condition. Wearable medical technology is already appearing, including the headsets that can keep a check on brain activity, cardiac monitoring chest bands, and remote diabetes monitors that can analyze glucose levels. In the near future, it is likely that there will be wearable technology that can monitor blood pressure changes and alert a doctor, and this century is likely to see the development of chips that can float in the bloodstream and monitor significant changes. 

\section{Remote Interaction}

The development of remote technology has also had an impact on the way that health care is delivered. Some clinics and hospitals have been installing routers in the homes of patients that make it possible for them to collect vital data on blood pressure, oxygen levels and glucose, which enables physicians to make adjustments to care regimes without having to call the patients into hospital, and to diagnose serious health changes and complications.

\begin{figure}[h]
\centering
\includegraphics[scale=.4]{remote_interaction}
\caption{Remote Interaction in healthcare}
\end{figure}

Some medical providers are also offering video consultations by using Skype or similar video conferencing systems. E-health consultations have the potential to dramatically reduce the cost and inconvenience  of in-patient treatment, allowing physicians to carry out assessment, make diagnoses and prescribe treatment without the patients leaving their home. Some doctors are even training to specialize in offering virtual consultations.

\section{The Power of the Smartphone}   

The smartphone is ubiquitous. It has been estimated that by 2020, around 80 percent of adults will own atleast one smartphone , making this technology ideal as a platform for the medical care of the future. As early as 2010, a campaign by the healthy moms healthy babies coalition arranged for pregnant women and new moms to receive regular text messages giving them useful information on how to care for themselves and their infants.

In the developing world, mobile technology is now regularly used for medical treatment. Physicians in Nepal introduced a mobile based antenatal care system that has drastically improved the speed and efficiency    of the care they receive, and in Tanzania, 125,000 women signed up for a mobile messaging service offering vital health information.

\section{The Human Touch}

The rapid development of health technology is facilitating the creation of health care practices that can drastically change the relationship between doctor and patient, while enabling patients to be more involved in the process of health care. However, at the same time, one of the challenges of the 21st century will be ensure that there remain a human element to be care that patients receive, and that amid the rapid take up of medical technology, physicians and other health care professionals will still have the meaning of interacting with their patients in the meaningful way. 

\section{Conclusion}

From wearable technology to remote consultations, the 21st century has already seen some significant development in the way that the health care is delivered and accessed, and this pace of change is likely to accelerate over the next few decades.

In the line with these developments, you may also want to stay updated with the latest news in the use of Computer based devices in the management and treatment of a wide range of physical and mental health issues.      

\end{document}