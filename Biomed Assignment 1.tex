\documentclass[12pt]{article}
\usepackage{graphicx}

\begin{document}
\begin{titlepage}
\begin{center}

\large{NATIONAL INSTITUTE OF TECHNOLOGY, RAIPUR}\\
[1.5mm]
\begin{figure}[h]
\centering
\includegraphics[scale=.5]{NITRaipur_Logo}
\end{figure}
\line(1,0){300}\\
[0.25in]
\huge{\bfseries 5 MEDICAL DEVICES}\\
[2mm]
\line(1,0){200}\\
[0.25in]
\large{\bfseries ASSIGNMENT 1}\\
{Basic Biomedical Engineering}\\
[0.75cm]
\large{Aditya Shrotriya}\\
{Roll number- 21111005}\\
{January 28, 2022}\\
[1cm]
\large{Under the supervision of:}\\
{Dr. Saurabh Gupta}

\end{center}
\end{titlepage}
\clearpage
\tableofcontents 
\clearpage

\section{Spinal Decompression Table}

\subsection{Introduction}
Patient who suffer from the chronic pain associated with bulging, degenerating, or herniated discs may benefit from the treatment using a spinal decompression table. This type of pain which can manifest as back or neck pain itself as well as associated pain in the arms and legs, may have already been treated by traditional traction methods or even by spinal surgery to limited improvement. In these cases, a spinal decompression table that uses computerized sensors to perform stretching actions on the spine and promoting healing can be uniquely effective.

\subsection{What is a Spinal Decompression Table?}
A spinal decompression table is the main tool used in non-surgical spinal decompression therapy. There are two main types of spinal decompression table: one with cable and pulley systems that create pull on the patient's body, and decompression tables that consist of an upper and lower body portion that move independently from one another. 

\begin{figure}[h]
\centering
\includegraphics[scale=0.5]{photo100}
\caption{Spinal Decompression Table}
\end{figure}

Patients are strapped to the table using a harness, with other props such as pillow used to keep the patient comfortable and the spine in the correct position for decompression. Once the patient is in place, the table program is enacted and the two parts of the table begin to pull apart from one another. The poundage of the pull depends on the type of decompression as well as the physicality of the patient and can range anywhere from just 5 pounds for a cervical decompression protocol to 100 pounds or more for lumbar decompression on the larger patient. 

The difference between spinal decompression tables and other methods of decompression like inversion is the technology involved. The best spinal decompression tables include sensors that indicate whether a patient's muscle are resisting the stretch being applied by the table. As the muscle resist, the table reduces the poundage of the pull until the muscle relax, then begins the pull again. By taking resistance into account, a decompression table is able to more effectively relieve pressure in the spine and ultimately provide relief.


\subsection{How a Spinal decompression table works?}
Spinal decompression tables are computerized technology to create negative intradiscal pressure in the spine. A decompression table has two parts who move independently of one another. During a start, a spinal decompression technician chooses a decompression program that is best suited to the patient's needs. The right program for an individual will depend the person's diagnosis as well as how they have responded to previous treatments.

As the decompression table stretches the spine, negative pressure is created within the spinal discs, which can result in the retraction or repositioning of the disc material,leading to pain relief. In addition the lower pressure within the disc can cause an influx of healing nutrients to the disc, to promote further relief even when the patient is not on the table. 

One of the hurdles to effective decompression with manual techniques is the fact that the body naturally resist the stretch, known as muscle guarding. With a decompression table, sensors can detect when the patient's muscle are guarding against the stretch and release the tension, ensuring that the the maximum decompression is accomplish. 

\clearpage

\section{Corneal Topography}

\subsection{Introduction}
Corneal topography is also known as photokeratoscopy or videokeratoscopy, is a non-invarsive medical imaging technique for mapping the anterior curvature of the cornea, the outer structure of the eye. Since the cornea is normally responsible for some 70 percent of the eye's refractive power, its topography is of critical importance in determining the quality of vision and corneal health.

The 3D map is therefore a valuable aid to the examining ophthalmologist or optometrist and can assist in the diagnosis and the treatment of number of conditions
\begin{itemize}
\item in planning cataract surgery and intraocular lens implantation.
\item in planning refractive surgery such as LASIK and evaluating its results.
\item in assessing the fit of contact lenses.
\end{itemize}
\begin{figure}[ht]
\centering
\includegraphics[scale=.8]{Corneal topography}
\caption{Corneal Topography showing stage}
\end{figure}
A development of keratoscopy, corneal topography extends the measurement range from the four points a few millimeters apart that is offered by keratometry to a grid of thousands of points covering the entire cornea. The procedure is carried out in seconds and is painless.

\subsection{How Corneal Topography works?}
The patient is seated facing the device, which is raised to eye level. One design consists of bowl containing an illuminated pattern, such as a series of concentric rings. Another type uses mechanically rotated arm bearing a light source. In either type, light is focused on the anterior surface of the patient's cornea and reflected back to a digital camera at the device.

The topology of the cornea is revealed by the shape taken by the reflected pattern. A computer provides a necessary analysis,typically determining the position and height of several thousand points across the cornea. The topographical map can be represented in a number of graphical formats, such as a sagittal map, which color-codes the steepness of the curvature according to its dioptric value.  
 
\clearpage
 
\section{Oxygen Therapy System}

\subsection{Introduction}
Oxygen Therapy refers to a treatment in which a patient in which a patient is provided supplemental oxygen. This therapy is given to those who are not able to get enough oxygen on their own. This can be happen to those with lung diseases like asthma, chronic obstructive pulmonary disease (COPD) and pneumonia or other health conditions like heart failure and sleep apnea. This therapy improves the blood oxygen levels of the patient and helps them feel better.
 
\subsection{What is Oxygen Therapy?}
Oxygen is an important gas that the human body needs to function and survive. It makes up about 21 percent of all the air you breathe. Your lungs absorb this gas and pass it over to your blood. Haemoglobin present in the blood then carries this oxygen to all your body cells.

If you have a lung disease, your lung may be inflamed or scarred and would be unable to pass  on enough of oxygen to your body. Low oxygen in blood is known as hypoxemia. On the other hand, hypoxia is more severe condition in which the whole body or a body tissue does not get enough of oxygen. Hypoxemia is the most common cause hypoxia. Tissue damage may occur due to lack of oxygen.

Your doctor may decide to give you Oxygen therapy if you have low blood levels. If you have low oxygen levels, you may show the following symptoms:
\begin{itemize}
\item Shortness of breath
\item Fast breathing
\item Increased heart rate and blood pressure
\item Tiredness
\item Confusion
\item Cyanosis (bluish discolouration of skin) 
\end{itemize}

\subsection{Types of Oxygen Therapy System}
Oxygen therapy is either given through the face masks or tubes inserted into the patient's nose  or windpipe (trachea). The following devices are used to store and deliver oxygen:
\begin{itemize}
\item Oxygen concentrator: Oxygen concentrators are devices that can take the air you breathe and concentrate the oxygen level in it to about 85 percent to 95 percent the normal oxygen level in air is 21 percent. This helps the patient's lung to take more oxygen. Oxygen concentrators can also be portable and used in home. Some concentrators send a continuous oxygen supply while some send pulse of oxygen.

\item Compressed gas cylinders: Compressed gas cylinders, as the name suggests, have oxygen compressed inside various sized containers or cylinders. The cylinder have 100 percent oxygen at a pressure of about 15169 kilo pascal and about 21 degree Celsius. Larger sized cylinders are used in hospitals and can supply directly to the medical gas pipelines. 
 
 \item Liquid oxygen: Oxygen takes up less space in liquid form. So, more oxygen can be stored at once. Liquid oxygen is prepared by cooling oxygen to a temperature of minus 300 degree Celsius. This method also provides 100 percent oxygen and can be used in Large-sized or portable cylinders.  
\end{itemize} 

\subsection{Side effects of Oxygen Therapy}
Normally, oxygen therapy is considered to be safe. However it may have some of the following side effects:
\begin{itemize}
\item Drowsiness
\item Headache, especially in the morning 
\item Feeling tired
\item Dry or bloody nose
\end{itemize}
Some side effects of hyperbaric oxygen therapy include claustrophobia, ear pressure, headache and fatigue.Long term exposure to high levels of oxygen may lead to oxygen toxicity in some people. 

\section{Fetal Monitor}

\subsection{Introduction}
A fetal monitor is a hand held ultrasound transducer used to detect the fetal heartbeat for prenatal care. It uses the doppler effect to effect to provide an audible simulation of the heart beat. Some models also display the heart rate in beat per minutes. Use of this monitor is sometimes known as Doppler auscultation. The Doppler fetal monitor is commonly reffered to simply as a doppler or fetal doppler. It may be classified as a form of doppler ultrasonography (although usually non technically graphy but rather sound generating). 

\begin{figure}[ht]
\centering 
\includegraphics[scale=.8]{BabyDoppler}
\caption{Fetal Heart Rate and Activity Monitor}
\end{figure}

Doppler fetal monitors provide information about the fetus similar to that provided by a fetal stethoscope. One advantage of the doppler fetal monitor over a fetal stethoscope is the electronic audio output, which allows people other than the user to hear the heartbeat. One disadvantage is the greater complexity and cost and the lower reliability of an electronic device.

The device was invented in 1958, originally intended for use by health care professionals, this device is becoming popular for personal use.

\subsection{Types of Fetal Doppler Monitor}

Doppler for home or hospital use differ in the following ways:
\begin{itemize}
\item Manufacturer: Popular manufacturers Baby doppler, Sonoline, Ultrasound technologies, Newman medical, Nicolet, Arjo-Huntleigh, and summit doppler (now copper surgical). 
\item Probe type: Waterproof or not. Waterproof probes are used for water births.
\item Probe Frequency: 2 MHz or 3 MHz probes. Most practitioners can find the heart rate with either probe. A 3 MHz probe is recommended to detect a heart rate in early pregnancy.
\item Heart Rate display: Some doppler automatically displays the heart rate on a build-in LCD, for other the fetal heart rate must be counted and timed by the practitioner.
\end{itemize}

\subsection{Conclusion}
A major advantage of being able to record and share the recording is that it can be emailed to a healthcare professional to be checked if there are any concerns about whether or not it is the fetus's heartrate and whether or not is normal. Typically, they work from about 12 weeks. 

In response to increasing home usage of clinical fetal doppler systems, the FDA issued a formal statement recommending against at-home use. System misuse and systems operating outside of intened range can produce thermal and non thermal effects on fetal tissue, including the possibility for over heating fetal tissue and introducing mechanical stress on the fetus due to carvitation, radiation force, and acoustic streaming. 
\clearpage

\section{Wound Management}

\subsection{Introduction}
Wound management is a treatment which includes wound healing, wound dressing and wound cleansing. The process is not not complex but also fragile, and it is susceptible to interruption of failure leading to the formation of non-healing chronic wounds. Factors that contribute to non healing chronic wounds are diabetes, venous or arterial disease, infection and metabolic deficiencies of old age.    

\subsection{Wound Healing}
Would healing refers to a living organism's replacement of damaged or destroyed tissue by newly produced tissue.

In undamaged skin, the epidermis and dermis form a protective barrier against the external environment. When the barrier is broken, a regulated sequence of biochemical events is set into motion to repair the damage. This process is divided into predictable phases: blood clotting, inflammation, tissue growth and tissue re-modeling. Blood clotting may be considered to be a part of the inflammation stage instead of a separate stage. 

Wound care encourages and speeds wound healing via cleaning and protection from re-injury or infection. Depending on each patient's needs, it can range from the simplest first aid to entire nursing specialities such as wound, ostomy, and continence nursing and burn centre care.

\subsection{Wound Dressing}   
Medical wound dressing to wound repair have undergone considerable research and development  in recent years. Scientists aim to develop wound dressing which have the following characteristics.
\begin{itemize}
\item Provide wound protection 
\item Remove excess exudate
\item Anti-microbial properties
\item Maintain a humid environment 
\item Have high permeability to oxygen
\item Easily removed from a wound site 
\end{itemize}
Cotton gauze dressing have been the standard of care, despite their dry properties that can adhere to wound surfaces and cause discomfort upon removal. Recent research has set out to improve cotton gauze dressing to bring them closer in line to achieve the modern wound dressing properties,by coating cotton gauze wound dressing with a nano-composite. These updated dressing prove increase water absorbency and improved antibacterial efficacy.

\subsection{Wound Cleansing} 
Dirt or dust on the surface of the wound, bacteria, tissue that has been died, and fluid from the wound my be cleaned. The evidence supporting the most effective technique is not clear and there is inefficient evidence to conduct whether cleaning wounds is beneficial for promoting healing or whether wound cleaning solutions are better than sterile water or saline solution to help venous leg ulcers heal. It is uncertain whether the choice of cleaning solution or method of application makes any difference to venous leg ulcer healing. 

\end{document}
